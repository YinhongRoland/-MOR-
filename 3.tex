\section{线性规划的单纯型算法}

\begin{frame}{\secname}
\begin{itemize}
    \item  图解法只能求解比较简单的线性规划形式,利用计算机求解虽然能够解决大规模的线性规划问题,但是掩盖了实际的求解方法和过程。为了更深入地学习线性规划和开发新的算法,必须要对线性规划的求解机理有一定的了解。
\item 本章在对线性规划的模型有一定认识后,从代数形式上继续考察线性规划问题。首先,给出线性规划相关的一些概念和最优解的性质;然后,在此基础上提出单纯形原理;在对单纯形法进行完整描述后,本章将对单纯形法提出了进一步的说明。通过本章的学习,读者应该对线性规划的单纯形的思想和步骤有较为清晰的理解。
\end{itemize}
\end{frame}

\subsection{线性规划的基本理论}
\begin{frame}{可行区域的几何结构}
考虑标准的线性规划问题:
$$
\min c^Tx
$$
$$
Ax=b,\quad x\ge 0
$$
用$R^n$ 表示n维的欧式空间,这里$\vx \in R^n$, $c\in R^n$, $b\in R^n$, $\vA \in R^{m\times n}$. 不妨设可行区域$\vD=\{x\in R^n \mid \vA \vx=b, x\ge 0\}\neq \Phi$,因此线性方程组      相容,总可以把多余方程去掉,使剩下的等式约束的系数向量线性无关,故可设$秩(\vA)=m$,$m<n$。

\end{frame}












\begin{frame}{\secname} 
  \begin{definition}[行秩 \& 列秩]
    \begin{itemize}
    \item      
      $\vA$的行向量组的秩,称为矩阵$\vA$的\red{行秩}。
    \item
      $\vA$的列向量组的秩,称为矩阵$\vA$的\red{列秩}。
    \end{itemize}      
  \end{definition}
  \pause 
  \begin{alertblock}{注}
    对于$m\times n$阶矩阵$\vA$,其行秩$\le m$,列秩$\le n$。
  \end{alertblock}
\end{frame}

\begin{frame}{\secname}
  
  \begin{definition}[阶梯型矩阵]
    若矩阵$\vA$满足
    \begin{itemize}
    \item[(1)] 零行在最下方;
    \item[(2)] 非零行首元的列标号随行标号的增加而严格递增,
    \end{itemize}
    则称$\vA$为\red{阶梯型矩阵}。
  \end{definition}
  \begin{example}
    $$
    \left(
    \begin{array}{rrrr}
      2&3&2&1\\
      0&5&2&-2\\
      0&0&3&2\\
      0&0&0&0
    \end{array}
    \right)
    $$
  \end{example}
  
\end{frame}


\begin{frame}{\secname}
  
  \begin{definition}[行简化阶梯型矩阵]
    若矩阵$\vA$满足
    \begin{itemize}
    \item[(1)] 它是阶梯型矩阵;
    \item[(2)] 非零首元为$1$;
    \item[(3)] 非零行首元所在的列除其自身外,其余元素全为零,
    \end{itemize}
    则称$\vA$为\red{行简化阶梯型矩阵}。
  \end{definition}
  \begin{example}
    $$
    \begin{pmatrix*}[r]
      1&2&0& 1&0&3\\
      0&0&1&-2&0&4\\
      0&0&0& 0&1&5\\
      0&0&0& 0&0&0
    \end{pmatrix*}
    $$
  \end{example}
  
\end{frame}


