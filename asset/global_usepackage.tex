\usepackage{amsmath,amsthm,amssymb,mathdots}
\usepackage{mathtools} %% pmatrix*
%% \usepackage{fourier}
\usepackage{multicol}
\usepackage{multirow}
% \usepackage[table]{xcolor}    
%\usepackage{fontspec}
\usepackage{bbding}
\usepackage{subfigure}
%\usepackage[most]{tcolorbox}
\newcounter{testexample}
\usepackage{xparse}
\usepackage{lipsum}
\usepackage[UTF8,noindent]{ctex}
\usepackage{extarrows}
\usepackage{array}
%% \usepackage{courier}
\usepackage{animate}
\usepackage{dcolumn}
\usepackage[ruled,vlined,linesnumbered]{algorithm2e}
\usepackage[]{hyperref}
\usepackage{pgf}
%% \usepackage{pgfplots}
\usepackage[framemethod=TikZ]{mdframed}
\usepackage{tikz}
\usetikzlibrary{calc,shapes.multipart,chains}
\usetikzlibrary{arrows,snakes,backgrounds,shapes,patterns,shadows}
\usetikzlibrary{matrix,fit,positioning,decorations.pathmorphing}
%% \usepackage{calligraphy}
%% \usepackage{matrixcells}
%% \usepackage{tikzamsmatrix}
\usepackage{color, colortbl}
\usepackage{cases}
\usepackage{enumerate}

\usepackage{listings}
\lstset{
        language=python,
        keywordstyle=\color{red},
        % frame=single,
        basicstyle=\ttfamily,
        commentstyle=\color{blue},
        breakindent=0pt,
        rulesepcolor=\color{red!20!green!20!blue!20},
        rulecolor=\color{black},
        tabsize=4,
        numbersep=5pt,
        breaklines=true,
        %backgroundcolor=\color{red!15},
        showstringspaces=false,
        showspaces=false,
        showtabs=false,
        extendedchars=false,
        escapeinside=``,
}
\tikzset{mystyle/.style n args={5}{
    rectangle,
    draw,
    fill=#1!50,
    % rounded corners,
    minimum height=#2cm,
    minimum width=#3cm,
    text width=#4cm,
    align=#5
  }}

\tikzset{mystyle1/.style n args={5}{
    rectangle,
    fill=#1!50,
    minimum height=#2cm,
    minimum width=#3cm,
    text width=#4cm,
    align=#5
  }}

\tikzset{mystyle2/.style n args={5}{
    rectangle,
    % draw,
    fill=#1!30,
    % rounded corners,
    minimum height=#2,
    minimum width=#3\textwidth,
    text width=#4\textwidth,
    align=#5
  }}


\tikzset{myarrow/.style n args={2}{
    ->,
    #1,
    line width=#2
  }}

\tikzset{list/.style={
rectangle split,
rectangle split parts=2,
rectangle split horizontal,
rectangle split part fill={red!30,blue!20},
rounded corners,
node distance=.5cm,
draw=black, thick,
minimum height=0.35cm,
text width=.5cm,
text centered,
}}

\tikzset{dlist/.style args={#1}{
rectangle split,
rectangle split parts=3,
rectangle split horizontal,
rectangle split part fill={blue!20,red!30,blue!20},
rounded corners,
node distance=#1mm,
draw=black, thick,
minimum height=0.35cm,
text centered,
}}

\tikzset{sentinel/.style args={#1}{
rectangle split,
rectangle split parts=3,
rectangle split horizontal,
rectangle split part fill={white,white,white},
rounded corners,
node distance=#1mm,
draw=black, thick,
minimum height=0.35cm,
text centered,
}}

\tikzset{square/.style args={#1}{
rectangle,draw,
fill=#1!30,
minimum height=.35cm,
minimum width=.35cm
}}

\usepackage{refcount}
\usepackage{multicol}
\newcounter{countitems}
\newcounter{nextitemizecount}
\newcommand{\setupcountitems}{%
        \stepcounter{nextitemizecount}%
        \setcounter{countitems}{0}%
        \preto\item{\stepcounter{countitems}}%
}
\makeatletter
\newcommand{\computecountitems}{%
        \edef\@currentlabel{\number\c@countitems}%
        \label{countitems@\number\numexpr\value{nextitemizecount}-1\relax}%
}
\newcommand{\nextitemizecount}{%
        \getrefnumber{countitems@\number\c@nextitemizecount}%
}
\newcommand{\previtemizecount}{%
        \getrefnumber{countitems@\number\numexpr\value{nextitemizecount}-1\relax}%
}
\makeatother    
\newenvironment{AutoMultiColItemize}{%
        \ifnumcomp{\nextitemizecount}{>}{3}{\begin{multicols}{2}}{}%
                \setupcountitems\begin{itemize}}%
                {\end{itemize}%
                \unskip\computecountitems\ifnumcomp{\previtemizecount}{>}{3}{\end{multicols}}{}}


%% If you build your own environment using array, you're on the safe side. I would extend an int%% ernal macro of amsmath using an optional argument.
%%
%% Advantages:
%%
%% It extends several matrix environments at the same time (matrix, pmatrix, bmatrix, Bmatrix, vmatrix, Vmatrix).
%%
%% The names and meanings of those environments remain (not apmatrix etc.)
%%
%% Spacing etc. is the same like in amsmath.
%%
%% You could do more than just insert a vertical line (use color and alignment, for instance right aligned columns because of minus signs).
%%
%% If you omit the optional argument, it acts exactly like the amsmath environment.
%%
%% Caution:
%%
%% Since you redefine an internal macro, it might not work if the original package changes its code. But amsmath.sty has not been changed for more than 10 years. If there's a change in the matrices later, you could adjust your own macro.
%% Code:
%% Here's the redefinition, just put it in your preamble after loading amsmath:
\makeatletter
\renewcommand*\env@matrix[1][*\c@MaxMatrixCols c]{%
  \hskip -\arraycolsep
  \let\@ifnextchar\new@ifnextchar
  \array{#1}}
\makeatother

%% Examples:

%% Simple augmented matrix:

%% \begin{pmatrix}[cc|c]
%%   1 & 2 & 3\\
%%   4 & 5 & 9
%% \end{pmatrix}
%% More complex use, with different alignment, spacing and color:

%% \begin{bmatrix}[*2cr@{\quad}|@{\quad}>{\color{red}}r]
%%   a & b & 1  &  4 \\
%%   c & d & -2 & -3
%% \end{bmatrix}
