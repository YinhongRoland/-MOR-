\section{对偶线性规划与灵敏度分析}

\subsection{变分法,哈密尔顿动力系统}

\begin{frame}{\subsecname}
\begin{definition}[拉格朗日量]
定义光滑函数
\begin{equation}\label{eq:lagrange}
    L: \R^n \times \R^n \to \R, \quad
    L = L(\vx, \vv), 
\end{equation}
称之为\red{拉格朗日量}. 
\end{definition}
\pause 

设$T > 0$,$\vx^0, \vx^1 \in \R^n$已知。
\begin{block}{变分法研究的基本问题(变分问题)}
求曲线$\vx^*: [0, 1] \to \R^n$以极小化泛函
\begin{equation}\label{eq:cvp}
    I[\vx(\cdot)] = \int_0^T L(\vx(t), \dot\vx(t))\,dt, 
\end{equation}
对所有满足条件$\vx(0)=\vx^0, \vx(T)=\vx^1$的函数$\vx(\cdot)$。
\end{block}
\end{frame}

% \subsubsection{欧拉-拉格朗日方程}
\begin{frame}{\subsecname}
    对拉格朗日量$L=L(\vx,\vv)$,将变量$\vx$看做是位置,将$\vv$看做是速度。记
    $$
    \frac{\partial L}{\partial x_i} = L_{x_i}, \quad
    \frac{\partial L}{\partial v_i} = L_{v_i}
    $$
    及
    $$
    \nabla_{\vx} L = 
    \begin{bmatrix}
    L_{x_1} & \cd & L_{x_n}
    \end{bmatrix}^T, \quad 
    \nabla_{\vv} L = 
    \begin{bmatrix}
    L_{v_1} & \cd & L_{v_n}
    \end{bmatrix}^T. 
    $$
    
    \pause 
    \begin{theorem}[欧拉-拉格朗日方程]
    设$\vx^*(\cdot)$为变分问题\eqref{eq:cvp}的解,则$\vx^*(\cdot)$必为\red{欧拉-拉格朗日方程}
    \begin{equation}\label{eq:EL}
        \frac{d}{dt} \left[\nabla_{\vv} L(\vx(t),\vv(t))\right] = \nabla_{\vx} L(\vx(t),\vv(t))
    \end{equation}
    的解。
    \end{theorem}
    \pause 
    \begin{block}{注}
    \begin{itemize}
    \item 若能求解欧拉-拉格朗日方程,则原变分问题的解(若存在)必在这些解中。
   \item 欧拉-拉格朗日方程是一个拟线性系统,它含$n$个二阶常微分方程,其中第$i$个方程可表示为
   \begin{equation}
       \frac{d}{dt} \left[L_{v_i}(\vx(t), \dot\vx(t))\right] = L_{x_i}(\vx(t), \dot\vx(t)).
   \end{equation}
    \end{itemize}

    \end{block} 
\end{frame}

\begin{frame}{\subsecname}
    \begin{block}{证明.}
    \begin{enumerate}
        \item 对任意光滑函数$\vy: [0,T]\to \R^n, \vy(0)=\vy(T)=\v0$,定义
        $$
        i(\tau) := I[\vx^*(\cdot) + \tau\vy(\cdot)], \quad \forall \tau \in \R.
        $$
        由于$\vx^*(\cdot)$为极小值点,故
        $$
        i(\tau) \ge i(0) = I[\vx^*(\cdot)] .
        $$
        即$i(\tau)$在$\tau=0$处取到极小值,从而
        $$
        i^\prime(0) = 0.
        $$
    \end{enumerate}
    \end{block}
\end{frame}

\begin{frame}{\subsecname}
    \begin{block}{证明.【续】}
    \begin{enumerate}[2]
        \item 计算$i^\prime(x)$. 由
        $$
        i(\tau) = \int_0^T L(\vx^*(t)+\tau\vy(t), \dot\vx^*(t)+\tau\dot\vy(t))\,dt
        $$
        可算得
        $$
        \begin{aligned}
        i^\prime(\tau) = \int_0^T&\left(\sum_{i=1}^nL_{x_i}\left(\vx^*(t)+\tau\vy(t), \dot\vx^*(t)+\tau\dot\vy(t)\right)y_i(t) + \right.\\
        &~~\left.\sum_{i=1}^nL_{v_i}\left(\vx^*(t)+\tau\vy(t), \dot\vx^*(t)+\tau\dot\vy(t)\right)\dot y_i(t)\right)\,dt
        \end{aligned}
        $$
        令$\tau=0$,则
        $$
        i^\prime(0) = \int_0^T\left(\sum_{i=1}^nL_{x_i}\left(\vx^*(t), \dot\vx^*(t)\right)y_i(t) +\sum_{i=1}^nL_{x_i}\left(\vx^*(t), \dot\vx^*(t)\right)y_i(t)\right)\,dt, 
        $$
        该式对任意光滑函数$\vy: [0,T]\to \R^n, \vy(0)=\vy(T)=\v0$皆成立。
     \end{enumerate}
    \end{block}
\end{frame}

\begin{frame}{\subsecname}
    \begin{block}{证明.【续】}
    \begin{enumerate}[3]
        \item  任取$1\le j \le n$并固定之,选取$\vy(\cdot)$,其分量为
        $$
        y_i(t) = \left\{
        \begin{array}{cc}
            \psi(t),  &  i=j,\\
            0,   & i\ne j,
        \end{array}
        \right.
        $$
        其中$\psi:[0,T]\to \R$为任意满足$\psi(0)=\psi(T)=0$的函数,则
        $$
        0 = \int_0^T L_{x_j}\left(\vx^*(t), \dot\vx^*(t))\psi(t) + L_{v_j}(\vx^*(t),\dot\vx^*(t)\right) \dot\psi(t)\,dt.
        $$
        利用分部积分及$\psi(0)=\psi(T)=0$可知
        $$
        0 = \int_0^T \left[L_{x_j}\left(\vx^*(t), \dot\vx^*(t)\right) - \frac{d}{dt}L_{v_j}\left(\vx^*(t),\dot\vx^*(t)\right)\right] \psi(t)\,dt.
        $$
        由$j$和$\psi(t)$的任意性可知,
        $$
        L_{x_j}\left(\vx^*(t), \dot\vx^*(t)\right) - \frac{d}{dt}L_{v_j}\left(\vx^*(t),\dot\vx^*(t)\right) = 0, \quad 1 \le j \le n. 
        $$
        定理得证。
    \end{enumerate}
    \end{block}
\end{frame}

% \subsubsection{哈密尔顿系统}
\begin{frame}{\subsecname}
\begin{definition}[广义动量]
对于给定的曲线$\vx(\cdot)$,定义
\begin{equation}\label{eq:p}
    \vp(t) := \nabla_{\vv} L(\vx(t),\vv(t)), \quad 0 \le t \le T,
\end{equation}
称之为广义动量(generalized momentum)。
\end{definition}

\begin{block}{重要假设}
给定$\vx,\vp\in\R^n$,方程
\begin{equation}
\vp = \nabla_{\vv} L(\vx,\vv)
\end{equation}
的解记为
\begin{equation}\label{eq:vxp}
\vv = \vv(\vx,\vp).
\end{equation}
\end{block}

\pause 
\begin{definition}[哈密尔顿量]
定义哈密尔顿量$H:\R^n\times \R^n \to \R$:
\begin{equation}\label{eq:hamilton}
    H(\vx, \vp) = \vp \cdot \vv(\vx, \vp) - L(\vx, \vv(\vx, \vp)),
\end{equation}
其中$\vv$由\eqref{eq:vxp}定义。
\end{definition}
\end{frame}

\begin{frame}{\subsecname}
    对哈密尔顿量$H:=H(\vx,\vp)$,记
    $$
    \frac{\partial H}{\partial x_i} = H_{x_i}, \quad
    \frac{\partial H}{\partial p_i} = H_{p_i}
    $$
    及
    $$
    \nabla_{\vx} H = 
    \begin{bmatrix}
    H_{x_1} & \cd & H_{x_n}
    \end{bmatrix}^T, \quad 
    \nabla_{\vp} H = 
    \begin{bmatrix}
    H_{p_1} & \cd & H_{p_n}
    \end{bmatrix}^T. 
    $$
    \begin{theorem}[哈密尔顿动力系统]
    设$\vx(\cdot)$为欧拉-拉格朗日方程的解,$\vp(\cdot)$由\eqref{eq:p}定义,则$(\vx(\cdot), \vp(\cdot))$为\red{哈密尔顿动力系统}
    \begin{equation}\label{eq:HDS}
        \left\{
        \begin{array}{rcr}
            \dot \vx(t) &=&  \nabla_{\vp} H(\vx(t), p(t)),   \\[.1in]
            \dot \vp(t) &=& -\nabla_{\vx} H(\vx(t), p(t)).
        \end{array}
        \right.
    \end{equation}
    另外,映射$t \mapsto H(\vx(t), \vp(t))$为常数。
    \end{theorem}
\end{frame}

\begin{frame}{\subsecname}
\begin{block}{证明.}
\begin{enumerate}
    \item 由$H(\vx,\vp)=\vp\cdot\vv(\vx,\vp)-L(\vx,\vv(\vx,\vp))$可知,
    $$
    \nabla_\vx H(\vx,\vp) = \vp\cdot \nabla_\vx \vv(\vx,\vp) - \nabla_\vx L(\vx,\vv(\vx,\vp)) - \nabla_\vv L(\vx,\vv(\vx,\vp)) \nabla_\vx \vv(\vx,\vp),
    $$
    再由$\vp=\nabla_\vv L(\vx,\vv)$可得
    \begin{equation}\label{eq:Hx}
        \nabla_\vx H(\vx,\vp) = -  \nabla_\vx L(\vx,\vv(\vx,\vp)) .
    \end{equation}
    注意到\red{$~\vp(t)=\nabla_\vv L(\vx(t),\dot\vx(t))$当且仅当$\dot\vx(t)=\vv(\vx(t),\vp(t))$}。
    因此,若$\vx(t)$是欧拉-拉格朗日方程的解,则
    $$
    \dot\vp(t) = \frac{d}{dt}\nabla_\vv L(\vx(t),\dot\vx(t)) 
    = \nabla_\vx  L(\vx(t),\dot\vx(t)) 
    = \nabla_\vx  L(\vx(t),\vv(\vx(t),\vp(t))).
    $$
    由\eqref{eq:Hx}可知,
    $$
    \dot\vp(t) = - \nabla_\vx H(\vx,\vp).
    $$
\end{enumerate}
\end{block}
\end{frame}

\begin{frame}{\subsecname}
\begin{block}{证明.【续】}
\begin{enumerate}[2]
    \item 由$H(\vx,\vp)=\vp\cdot\vv(\vx,\vp)-L(\vx,\vv(\vx,\vp))$可知,
    $$
    \nabla_\vp H(\vx,\vp) = \vv(\vx,\vp) + \vp \cdot \nabla_\vp \vv - \nabla_\vv L(\vx,\vv(\vx,\vp)) \nabla_\vp \vv(\vx,\vp).
    $$
    再由$\vp=\nabla_\vv L(\vx,\vv)$可得
    \begin{equation}\label{eq:Hp}
        \nabla_\vp H(\vx,\vp) = \vv(\vx,\vp).
    \end{equation}
    注意到\red{$~\vp(t)=\nabla_\vv L(\vx(t),\dot\vx(t))$当且仅当$\dot\vx(t)=\vv(\vx(t),\vp(t))$}。
    由\eqref{eq:Hp}可知,
    $$
    \dot\vx(t) = \nabla_\vp H(\vx(t),\vp(t)).
    $$
    \pause 
    \item 由
    $$
    \begin{aligned}
    \frac{d}{dt} H(\vx(t),\vp(t)) &= \nabla_\vx H(\vx,\vp) \cdot \dot\vx(t)+\nabla_\vp H(\vx,\vp) \cdot \dot\vp(t)\\
    &= -\dot\vp(t) \cdot \dot\vx(t)+\dot\vx(t) \cdot \dot\vp(t)\\
    & = 0
    \end{aligned}
    $$
    可知映射$t \mapsto H(\vx(t), \vp(t))$为常数。
\end{enumerate}
\end{block}
\end{frame}

\begin{frame}{\subsecname}
\small 
    \begin{example}[物理实例]
     定义拉格朗日量为
     $$
     L(\vx,\vv) = \frac{m|\vv|^2}{2} - V(\vx),
     $$
     可解释为动能(kinetic energy)减去势能(potential energy)。此时,
     $$
     \nabla_\vx L(\vx,\vv) = -\nabla V(\vx), ~~
     \nabla_\vv L(\vx,\vv) = m\vv ~~\text{\red{(动量)}}.
     $$
     由此可知,
     \begin{itemize}
         \item 欧拉-拉格朗日方程为
         $$
         m\ddot \vx (t) = m \dot\vv(t) = -\nabla V(\vx) = \vF,
         $$
         此即\red{牛顿第二运动定律}。
         \item 注意$\vp = \nabla_\vv L(\vx,\vv) = m\vv$为动量,则哈密尔顿量为
         $$
         H(\vx,\vp) = \vp \cdot \vv - L(\vx, \vv(\vx, \vp))
         = \vp \cdot \frac{\vp}{m} - \frac{m|\vp|^2}{2m^2} + V(\vx)
         = \frac{|\vp|^2}{2m} + V(\vx)
         $$
         此即动能与势能之和,亦即总能量(\red{由热力学第一定律,封闭系统中总能量不变})。对应的哈密尔顿系统为
         $$
         \left\{
         \begin{array}{rcl}
              \dot \vx(t)&=& \frac{\vp(t)}{m},  \\[.1in]
              \dot \vp(t)&=& -\nabla V(\vx).
         \end{array}
         \right.
         $$
     \end{itemize}
    \end{example}
\end{frame}

\subsection{拉格朗日乘子法综述}

\subsection{庞特里亚金极大值原理(固定时间,自由端点)}
 
\begin{frame}{\subsecname}
\begin{itemize}
    \item $\vA\subset \R^m$,$\vf: \R^n\times \vA \to \R^n$和$\vx^0\in\R^n$。
    \item \red{容许控制集合}
    $$
    \cA = \left\{\valpha(\cdot):[0,\infty]\mapsto \vA~|~\valpha(\cdot)\text{可测}\right\}.
    $$
    \item \red{常微分方程组}
    \begin{equation}\label{eq:ode_sec4}
        \left\{
        \begin{array}{rcll}
             \dot \vx(t)&=& \vf(\vx(t),\valpha(t)), & t > 0  \\
             \vx(t)&=&\vx^0. & 
        \end{array}
        \right.
    \end{equation}
    其中$\valpha(\cdot)\in \cA$给定。
    \item \red{收益泛函}
    \begin{equation}\label{eq:payoff_sec4}
        P[\valpha(\cdot)] = \int_0^T r(\vx(t),\valpha(t))\,dt + g(\vx(T)),
    \end{equation}
    其中终止时间$T>0$、运行收益$r:\R^n \times A\to \R$、终止收益$g:\R^n\to \R$给定。
    \item \red{最优控制问题:} 求$\valpha^*(\cdot)$是的
    $$
    P[\valpha^*(\cdot)] = \max_{\valpha(\cdot)\in \cA} P[\valpha(\cdot)].
    $$
    \item \red{哈密尔顿量} 
    \begin{equation}\label{eq:hamilton_control}
        H(\vx,\vp,\valpha) = \vf(\vx,\valpha)\cdot \vp + r(\vx,\valpha), \quad \vx,\vp\in \R^n, \valpha\in \cA.
    \end{equation}
    % \item \blue{庞特里亚金极大值原理告诉我们:存在一个函数$\vp^*(\cdot)$和最优轨迹$\vx^*(\cdot)$满足一个哈密尔顿系统。}
\end{itemize}
 
    
\end{frame}

\begin{frame}{\subsecname}
    \begin{theorem}[庞特里亚金极大值原理]
    设$\valpha^*$为\eqref{eq:ode_sec4}和\eqref{eq:payoff_sec4}的最优控制,$\vx^*$为对应的最优轨迹,则存在一个共态(costate)函数$\vp^*:[0,T]\to \R^n$使得$(\vx^*, \vp^*)$是如下哈密尔顿系统的解:
    \begin{numcases}{}
        \dot \vx(t)=\nabla_\vp H(\vx(t), \vp(t), \valpha(t)), \label{eq:pmp1_hs1} \\[.1in]
        \dot \vp(t)=-\nabla_\vx H(\vx(t), \vp(t), \valpha(t)). \label{eq:pmp1_hs2}
    \end{numcases}
    且
    \begin{equation}\label{eq:pmp1_mp}
        H(\vx^*(t), \vp^*(t), \valpha^*(t)) = \max_{\valpha(\cdot)\in \cA} H(\vx^*(t), \vp^*(t), \valpha(t)).  
    \end{equation}
    另外,
    \begin{itemize}
        \item 
        \begin{equation}
         H(\vx^*(t), \vp^*(t), \valpha^*(t)) \equiv C, ~~ \forall t \in [0,T].
        \end{equation}
        \item 
        终止条件满足
    \begin{equation}\label{eq:pmp1_transversality}
        \vp^*(T) = \nabla g(\vx^*(T)).
    \end{equation}
    \end{itemize}
 
    \end{theorem}
\end{frame}

\begin{frame}{\subsecname}
    \begin{block}{注}
    \begin{itemize}
        \item \eqref{eq:pmp1_hs2}为伴随(adjoint)方程,而\eqref{eq:pmp1_mp}为极大值原理;\eqref{eq:pmp1_hs1}和\eqref{eq:pmp1_hs2}共同构成哈密尔顿系统;\eqref{eq:pmp1_transversality}称为横截条件,其重要性将在后面阐述。
        \item 更具体地,
        \begin{itemize}
        \item \eqref{eq:pmp1_hs1}可表示为
        $$
        \dot \vx(t)=\vf(\vx,\valpha),
        $$
        即\eqref{eq:ode_sec4}中的动量方程。\\[.1in]
        \item \eqref{eq:pmp1_hs2}可表示为
        $$
        \dot \vp(t)= - \nabla_{\vx} \vf(\vx(t),\valpha(t))^T \vp(t) - \nabla_\vx r(\vx(t),\valpha(t)),
        $$
        其中$\nabla_{\vx} \vf$为Jacobi矩阵,即
        $$
        \nabla_{\vx} \vf(\vx,\valpha) =
        \begin{bmatrix}
        f_{x_1}^1 & f_{x_2}^1 & \cd & f_{x_n}^1  \\[.1in]
        f_{x_1}^2 & f_{x_2}^1 & \cd & f_{x_n}^2  \\[.1in]
        \vdots & \vdots & \dd & \vdots  \\[.1in]
        f_{x_1}^n & f_{x_2}^n & \cd & f_{x_n}^n  
        \end{bmatrix}
        $$
        \end{itemize}
    \end{itemize}
    \end{block}
\end{frame}

\begin{frame}{\subsecname}
\begin{example}[经济消费的简单控制问题]
 记$x(t)$为$t$时刻的产出,$\alpha(t)$为$t$时刻再投资的产出份额。
 \begin{itemize}
 \item 动力方程 
 $$
 \left\{
 \begin{array}{lc}
      \dot x(t)=\alpha(t)x(t), & t \in (0, T], \\[.05in]
      x(0) = x^0. & 
 \end{array}
 \right.
 $$
 其中$x:\R\to \R, \alpha:\R\to A=[0,1]\subset \R, x^0>0$。\\[.1in]
 \item 收益泛函 
 $$
 P[\alpha(\cdot)] = \int_0^T (1-\alpha(t))x(t)\,dt
 $$
 表示总消费。
 \\[.1in]
 \item 最优控制问题:求$\alpha^*(\cdot)$使得
 $$
 P[\alpha(\cdot)] = \max_{\alpha(\cdot)\in \cA} P[\alpha(\cdot)] 
 $$
 \end{itemize} 
\end{example}
\end{frame}

\begin{frame}{\subsecname}
\begin{block}{解}
对于该问题,
\begin{itemize}
\item \red{参数设置}
$$
m=n=1, ~
f(x,\alpha)=\alpha x, ~
r(x,\alpha)=(1-\alpha)x, ~
g(x) = 0.
$$
\item \red{哈密尔顿量}
$$
H(x,p,\alpha) = f(x,\alpha) p + r(x,\alpha) = \alpha p x + (1-\alpha)x.
$$
\item \red{哈密尔顿系统}
\begin{equation}
    \left\{
    \begin{array}{ll}
        \dot x(t) %=  H_p 
        = x(t)\alpha(t),  &  t \in (0,T], \\[.1in]
        \dot p(t) %= -H_x 
        =-1-\alpha(t)(p(t)-1),  &  t \in [0,T). 
    \end{array}
    \right.
\end{equation}
其中 
\begin{equation}
\left\{
\begin{array}{l}
    x(0) = x^0,  \\[0.05in]
    p(T) %= \nabla_x g(x(T)) 
    = 0.
\end{array}
\right.
\end{equation}
\item \red{极大值原理}
$$
H(x(t),p(t),\alpha(t)) = \max_{a\in [0,1]} \left\{a p(t)x(t)  + (1-a)x(t)\right\}. 
$$
\end{itemize}
\end{block}
\end{frame}

\begin{frame}{\subsecname}
\begin{block}{解【续】}
\begin{enumerate}
\item \blue{确定$\alpha(t)$与$p(t)$的关系}
\item[] 
由极大值原理以及$x(t) >0$,在$t$时刻应选取$\alpha(t)$以极大化$H(x,p,a)$,即
$$
\alpha(t) = \argmax_{a(t) \in [0,1]} H(x,p,a) = \argmax_{a(t) \in [0,1]} a(t)(p(t)-1). 
$$
因此,
\begin{equation}\label{eq:alpha_p_ex1}
\alpha(t) = \left\{
\begin{array}{cl}
    1, &  p(t) > 1, \\[.05in]
    0, &  p(t) \le 1.
\end{array}
\right.
\end{equation}
\end{enumerate}
\end{block}
\end{frame}

\begin{frame}{\subsecname}
\small 
\begin{block}{解【续】}
\begin{enumerate}[2]
\item \blue{确定共态$p^*(\cdot)$}
\item[] 伴随方程可表示为
\begin{equation}\label{eq:adj_ex1}
    \left\{
    \begin{array}{ll}
        \dot p(t) = -1-\alpha(t)(p(t)-1),  &  t \in [0,T).  \\[0.05in]
        p(T) = 0, & 
    \end{array}
    \right.
\end{equation} \pause 
由$p(T)=0$及$p(\cdot)$的连续性可知,当$t$靠近$T$时,$p(t) \le 1$,从而$\alpha(t)=0$。此时,\eqref{eq:adj_ex1}可改写为
\begin{equation}\label{eq:adj1_ex1}
    \left\{
    \begin{array}{ll}
        \dot p(t) = -1,  &  t \text{靠近} T.  \\[0.05in]
        p(T) = 0, & 
    \end{array}
    \right.
\end{equation}
其解为$p(t)=T-t$。\pause 
注意到当$t\le T-1$时,$p(t) > 1$,从而$\alpha(t) = 1$,因此
$$
p(t) = T-t, ~~ t \in[T-1, T). 
$$
\pause 
当$t\le T-1$时,\eqref{eq:adj_ex1}可改写为
\begin{equation}\label{eq:adj2_ex1}
    \left\{
    \begin{array}{ll}
        \dot p(t) = - p(t),  &  t \in[0, T-1].  \\[0.05in]
        p(T-1) = 1, & 
    \end{array}
    \right.
\end{equation}
其解为$p(t) = e^{T-1-t}$。综上所述,共态$p^*(\cdot)$可确定为
$$
p^*(t) = \left\{
\begin{array}{ll}
e^{T-1-t}, & t \in[0, T-1], \\[0.05in]
T-t, & t \in[T-1, T). 
\end{array}
\right.
$$
\end{enumerate}
\end{block}
\end{frame}

\begin{frame}{\subsecname}
\small 
\begin{block}{解【续】}

\begin{enumerate} 
\setcounter{enumi}{2}
\item \blue{确定最优控制$\alpha^*(\cdot)$}
\item[]  由\eqref{eq:alpha_p_ex1}可知,最优控制$\alpha^*(\cdot)$可确定为
$$
\alpha^*(t) = \left\{
\begin{array}{cl}
    1, &  t \in[0, t^*), \\[.04in]
    0, &  t \in[t^*, T],
\end{array}
\right.
$$
其中$t^*=T-1$为最佳切换时间。\pause \\[.1in]
\item \blue{确定最优轨迹$\vx^*(\cdot)$}
\item[] 当$t\in[0, T-1]$时,动力方程为
$$
\left\{
\begin{array}{ll}
    \dot x(t) = x(t),  &  t \in (0, T-1], \\[.04in]
     x(0) = x^0, & 
\end{array}
\right.
$$
其解为$x(t) = x^0 e^t$。当$t\in[T-1, T]$时,动力方程为
$$
\left\{
\begin{array}{ll}
    \dot x(t) = 0,  &  t \in (T-1, T], \\[.04in]
     x(0) = x^0 e^{T-1}, & 
\end{array}
\right.
$$
其解为$x(t) = x^0 e^{T-1}$。综上所述,最优轨迹$x^*(\cdot)$可确定为
$$
x^*(t) = \left\{
\begin{array}{cl}
    x^0 e^t, &  t \in[0, T-1), \\[.04in]
    x^0 e^{T-1}, &  t \in[T-1, T].
\end{array}
\right.
$$
\end{enumerate}
\end{block}
\end{frame}

\begin{frame}{\subsecname}
\begin{example}[线性二次型调节器(Linear-Quadratic Regulator, LQR)]
 \begin{itemize}
 \item \red{动力方程}
 $$
 \left\{
 \begin{array}{lc}
      \dot x(t)=x(t)+\alpha(t), & t \in (0, T], \\[.05in]
      x(0) = x^0. & 
 \end{array}
 \right.
 $$
 其中$x:\R\to \R, \alpha:\R\to A=\R$。\\[.1in]
 \item \red{代价泛函}
 $$
 C[\alpha(\cdot)] = \int_0^T \left[x(t)^2+\alpha(t)^2\right]\,dt
 $$
 \item \red{收益泛函}
 $$
 P[\alpha(\cdot)] = -\int_0^T \left[x(t)^2+\alpha(t)^2\right]\,dt
 $$
 \\[.1in]
 \item \red{最优控制问题}:求$\alpha^*(\cdot)$使得\blue{代价最小},即
 $$
 C[\alpha^*(\cdot)] = \min_{\alpha(\cdot)\in \cA} C[\alpha(\cdot)],
 $$
 或\blue{收益最大},即
 $$
 P[\alpha^*(\cdot)] = \max_{\alpha(\cdot)\in \cA} P[\alpha(\cdot)],
 $$
 \end{itemize} 
\end{example}
\end{frame}


\begin{frame}{\subsecname}
\begin{block}{解}
对于该问题,
\begin{itemize}
\item \red{参数设置}
$$
m=n=1, ~
f(x,\alpha)=x+\alpha, ~
r(x,\alpha)=-x^2-\alpha^2, ~
g(x) = 0.
$$
\item \red{哈密尔顿量}
$$
H(x,p,\alpha) = f(x,\alpha) p + r(x,\alpha) = (x+\alpha) p(t) - (x^2+\alpha^2).
$$
\item \red{哈密尔顿系统}
\begin{equation}
    \left\{
    \begin{array}{ll}
        \dot x(t) %=  H_p 
        = x(t)+\alpha(t),  &  t \in (0,T], \\[.1in]
        \dot p(t) %= -H_x 
        =2x(t)-p(t),  &  t \in [0,T). 
    \end{array}
    \right.
\end{equation}
其中 
\begin{equation}
\left\{
\begin{array}{l}
    x(0) = x^0,  \\[0.05in]
    p(T) %= \nabla_x g(x(T)) 
    = 0.
\end{array}
\right.
\end{equation}
\item \red{极大值原理}
$$
H(x(t),p(t),\alpha(t)) = \max_{a\in \R} \left\{(x(t)+a) p - (x^2(t)+a^2)\right\}. 
$$
\end{itemize}
\end{block}
\end{frame}

\begin{frame}{\subsecname}
\begin{block}{解【续】}
\begin{enumerate}
\item \blue{确定$\alpha(t)$与$p(t)$的关系}
\item[] 
由极大值原理,在$t$时刻应选取$\alpha(t)$以极大化$H(x,p,a)$,即
$$
\alpha(t)  = \argmax_{a \in [0,1]} \left\{a  p(t)-a^2\right\}. 
$$
对$a  p(t)-a^2$关于$a$求导并置其为零,得
$$
p(t) - 2a = 0, 
$$
由此可得$\alpha(t)$与$p(t)$的关系为
\begin{equation}\label{eq:alpha_p_ex2}
\alpha(t) = \frac{p(t)}{2}. 
\end{equation}
\end{enumerate}
\end{block}
\end{frame}

\begin{frame}{\subsecname}
\small 
\begin{block}{解【续】}
\begin{enumerate}[2]
\item \blue{确定轨迹$x(\cdot)$和共态$p(\cdot)$的关系}
\item[] 动力方程为
$$
\left\{
\begin{array}{ll}
    \dot x(t) = x(t) + \frac{p(t)}{2},  &  t \in (0, T-1], \\[.04in]
     x(0) = x^0, & 
\end{array}
\right.
$$
\item[] 伴随方程为
\begin{equation}\label{eq:adj_ex2}
    \left\{
    \begin{array}{ll}
        \dot p(t) = 2x(t)-p(t),  &  t \in [0,T).  \\[0.05in]
        p(T) = 0, & 
    \end{array}
    \right.
\end{equation} \pause
\item[] 哈米尔顿系统的矩阵形式为
\begin{equation}\label{eq:hds_ex2}
\begin{bmatrix}
\dot x(t) \\[0.05in]
\dot p(t)
\end{bmatrix}
= 
\underbrace{
\begin{bmatrix*}[r]
1 & \frac12\\[0.05in]
2 & -1
\end{bmatrix*}
}_{\vM}
\begin{bmatrix}
 x(t)\\[0.05in]
 p(t)
\end{bmatrix}
\end{equation}
 其解为
 $$
 \begin{bmatrix}
 x(t)\\[0.05in]
 p(t)
\end{bmatrix}
 = e^{t\vM} 
 \begin{bmatrix}
 x^0\\[0.05in]
 p^0
\end{bmatrix}. 
 $$
 \red{注意,由于$x^0$已知,还需选择$p^0$使得$p(T)=0$。}
\end{enumerate}
\end{block}
\end{frame}

\begin{frame}{\subsecname}
\small 
\begin{block}{解【续】}

\begin{enumerate} 
\setcounter{enumi}{2}
\item \blue{反馈控制(feedback control)}
\item[]  令$\alpha(t)=c(t)x(t)$,则
$
\ds \frac{p(t)}{2}=\alpha(t)=c(t)x(t) ~\Rightarrow~
c(t) = \frac{p(t)}{2x(t)}
$。
定义
\begin{equation}\label{eq:dt}
    y(t) = \frac{p(t)}{x(t)}
\end{equation}
则$c(t)=\frac12 y(t)$。对\eqref{eq:dt}两边关于$t$求导可得
$$
\dot y(t) = \frac{\dot p(t)x(t)-p(t)\dot x(t)}{x^2(t)} 
= \frac{\dot p(t)}{x(t)} - \frac{p(t)\dot x(t)}{{x^2(t)} }
$$
再由哈密尔顿系统\eqref{eq:hds_ex2}可得
$$
\dot y = \frac{2x-p}{x} - \frac{p}{x^2} (x+\frac{p}{2})
= 2 - \frac{2p}{x} - \frac{p^2}{2x^2} = 2-2y-\frac{y^2}{2}. 
$$
因$p(T)=0$,故$y(T)=0$。综上,我们推导出了具有终止条件的一阶常微分方程:
\begin{equation}\label{eq:raccati}
    \left\{
    \begin{array}{ll}
        \dot y(t) = 2-2y-\frac{y^2}{2},  &  t \in [0,T).  \\[0.05in]
        y(T) = 0, & 
    \end{array}
    \right.
\end{equation}
此即Raccati方程。\red{如果求出$y(t)$,则最优控制可表示为$\alpha^*(t)=\frac12 y(t)x(t)$。}
\end{enumerate}
\end{block}
\end{frame}

\begin{frame}{\subsecname}
\small 
\begin{block}{解【续】}

\begin{enumerate} 
\setcounter{enumi}{3}
\item \blue{求解Raccati方程}
\item[]  用积分法
$$
\int \frac{dy}{-\frac{y^2}{2}-2y+2} = \int \,dt 
~~\Longleftrightarrow~~
\int \frac{dy}{y^2+4y-4} = -2 t 
~~\Longleftrightarrow~~
\int \frac{dy}{(y+2)^2-8} = -2 t 
$$
令$z=y+2$,则上式可改写为
$$
\int \frac{dz}{z^2-8} = -2 t 
$$
由
$$
\frac1{z^2-8} = \frac1{4\sqrt2} \left(\frac{1}{z-2\sqrt2}-\frac{1}{z+2\sqrt2}\right)
$$
可知
$$
\ln \left|\frac{z-2\sqrt2}{z+2\sqrt2}\right| = -8\sqrt2 t + C 
% ~~\Longrightarrow~~
% \left|\frac{z-2\sqrt2}{z+2\sqrt2}\right| = Ce^{-8\sqrt 2t}
~~\xLongrightarrow[]{z(T)=2}~~
\left|\frac{z-2\sqrt2}{z+2\sqrt2}\right| = c e^{8\sqrt 2(T-t)} ~(c=\frac{\sqrt2-1}{\sqrt2+1}), 
$$
解得
$$
z(t) = 
$$
%

\end{enumerate}
\end{block}
\end{frame}

\subsection{庞特里亚金极大值原理(自由时间,固定端点)}
\begin{frame}{\subsecname}
\begin{itemize}
    \item $\vA\subset \R^m$,$\vf: \R^n\times \vA \to \R^n$和$\vx^0, \vx^1\in\R^n$。
    \item \red{容许控制集合}
    $$
    \cA = \left\{\valpha(\cdot):[0,\infty]\mapsto A~|~\valpha(\cdot)\text{可测}\right\}.
    $$
    \item \red{常微分方程组}
    \begin{equation}\label{eq:ode_case2}
        \left\{
        \begin{array}{ll}
             \dot \vx(t)= \vf(\vx(t),\valpha(t)), & t \in (0, \tau)  \\[.1in]
             \vx(0)=\vx^0, ~  \vx(\tau)=\vx^1, &   
        \end{array}
        \right.
    \end{equation}
    其中$\tau:=\tau[\valpha(\cdot)]\le \infty$表示轨迹$\vx(t)$第一次到达\blue{给定目标点$\vx^1$}的时间。
    \item \red{收益泛函}
    \begin{equation}\label{eq:payoff_case2}
        P[\valpha(\cdot)] = \int_0^\tau r(\vx(t),\valpha(t))\,dt,
    \end{equation}
    其中$r:\R^n \times A\to \R$为给定的运行收益。
    \item \red{最优控制问题:} 求$\valpha^*(\cdot)$是的
    $$
    P[\valpha^*(\cdot)] = \max_{\valpha(\cdot)\in \cA} P[\valpha(\cdot)].
    $$
    \item \red{哈密尔顿量}:
    $$
    H(\vx,\vp,\valpha) = \vf(\vx,\valpha)\cdot \vp + r(\vx,\valpha), \quad \vx,\vp\in \R^n, \valpha\in \cA.
    $$
\end{itemize}
\end{frame}

\begin{frame}{\subsecname}
    \begin{theorem}[庞特里亚金极大值原理]
    设$\valpha^*$为\eqref{eq:ode_sec4}和\eqref{eq:payoff_sec4}的最优控制,$\vx^*$为对应的最优轨迹,则存在一个共态(costate)函数$\vp^*:[0,\tau^*]\to \R^n$使得$(\vx^*, \vp^*)$是哈密尔顿系统
    \begin{numcases}{}
        \dot \vx(t)=\nabla_\vp H(\vx(t), \vp(t), \valpha(t)), \label{eq:hds_1} \\[.1in]
        \dot \vp(t)=-\nabla_\vx H(\vx(t), \vp(t), \valpha(t)). \label{eq:hds_2}
    \end{numcases}
    的解,且
    \begin{equation}\label{eq:max}
        H(\vx^*(t), \vp^*(t), \valpha^*(t)) = \max_{\valpha(\cdot)\in \cA} H(\vx^*(t), \vp^*(t), \valpha(t)), \quad t\in [0,\tau^*].
    \end{equation}
    另外,
    \begin{equation*}
        H(\vx^*(t), \vp^*(t), \valpha^*(t)) \equiv 0, \quad \forall t\in [0,\tau^*].
    \end{equation*}
    其中$\tau^*$为轨迹$\vx^*(t)$第一次到达目标点$\vx^1$的时间。
    % 称$\vx^*(\cdot)$为最优控制系统的状态(state),而$\vp^*(\cdot)$为共态(costate)。
    \end{theorem}
\end{frame}
 

\begin{frame}{\subsecname}
\begin{example}[线性时间最优控制]
\begin{itemize}
\item 动力方程
 $$
 \left\{
 \begin{aligned}
      \dot \vx(t)&=\vM\vx(t)+\vN\valpha(t), \\
      \vx(0)&=\vx^0. 
 \end{aligned}
 \right.
 $$
 其中$\vx:\R^n\to\R^n,\valpha:\R^n\to\vA=\R^m,\vM\in\R^{n\times n}, \vN\in\R^{n\times m}$。\\[.1in]
 \item 收益泛函
 $$
 P[\valpha(\cdot)] = -\int_0^\tau 1\,dt = -\tau,
 $$
 其中$\tau$为轨迹第一次到达目标点$\vx^1=\v0$的时间。\\[.1in]
 \item 最优控制问题:求最优控制$\valpha^*$使得
 $$
 P[\valpha^*(\cdot)] = \max_{\valpha(\cdot)\in\cA} P[\valpha(\cdot)].
 $$
\end{itemize}
\end{example}
    
\end{frame}

\begin{frame}{\subsecname}
\begin{block}{解}
此时$r=-1$,故哈密尔顿量为
 $$
 H(\vx,\vp,\valpha) = \vf\cdot \vp + r = (\vM\vx+\vN\valpha) \cdot \vp - 1.
 $$
 对应的哈密尔顿系统为
 $$
 \left\{
 \begin{aligned}
      \dot \vx(t) &= \vM\vx+\vN\valpha, \\
      \dot \vp(t) &=-\vM^T\vp 
 \end{aligned}
 \right.
 $$ 
\end{block}
   
\end{frame}

\begin{frame}{\subsecname}
\begin{example}[月球着陆]
对于一艘飞船,在$t$时刻,其高度为$h(t)$,速度为$v(t)$,质量为$m(t)$(随着燃料的使用发生变化),推力为$\alpha(t)$(控制变量)。
\begin{itemize}
 \item 动力方程 
 $$
 \left\{
 \begin{array}{ll}
      \dot h(t)=v(t), & h(0)=h_0>0\\[.05in]
      \dot v(t)=-g+\frac{\alpha(t)}{m(t)}, & v(0)=v_0\\[.05in]
      \dot m(t)=-k\alpha(t), & m(0)=m_0>0,
 \end{array}
 \right.
 $$
 其中$h,v,m:\R\to \R, \alpha:\R\to A=[0,1]\subset \R$。\\[.1in]
 \item 最优控制问题:由于目标是安全着陆,在着陆时刻$\tau$(首次使得$h(\tau)=v(\tau)=0$的时间),希望剩余燃料$m(\tau)$越多越好。 由于$\alpha=-\frac{\dot m}{k}$,问题等价于在着陆之前最小化总推力。故收益函数可定义为
 $$
 P[\alpha(\cdot)] = -\int_0^\tau \alpha(t) \,dt = \frac{m(\tau)-m_0}{k}.
 $$ 
 \end{itemize} 
\end{example}
\end{frame}
\subsection{庞特里亚金极大值原理(含横截条件)}

% \subsection{更多例子}

\subsection{庞特里亚金极大值原理(含状态约束)}

% \subsection{更多应用}

\begin{frame}{\subsecname}
  
   
  
\end{frame}



