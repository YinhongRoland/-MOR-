\section{存储论}

\begin{frame}{\secname}
\begin{itemize}
    \item 存储论主要考虑两个基本问题:供应量的多少和什么时候供应,即量和期的问题。存储论的基本方法是将一个实际存储问题归结为一种数学模型,然后通过费用分析求出最佳的期和量的数值,即选择合理的存储策略,使相应问题的总费用最小。
    \item 本章所介绍的存储问题,模型并不复杂,原理也很容易掌握,应用这些原理可以从一个方面改善企业的经营管理,以达到节约资金,获得更多利润的目的.
\end{itemize}
\end{frame}

\subsection{存储论的基本概念}

\begin{frame}{\subsecname}
工厂为了满足生产,必须要储存一些原料,把必需贮存的一些物资简称存储, 生产时从存储中取出一定数量的原料,使存储减少,当生产不断进行,存储不断减少,到一定时候必须对存储给以补充,否则存储用完了,生产就无法进,并且,生产是为需求而生产,所以,一般地说,存储量是因需求而减少,因补充而增加。
\end{frame}