\section{整数规划}

\subsection{定义}
\begin{frame}{\subsecname}
本文研究的博弈模型为:
    \begin{itemize}
        \item 两个人的零和博弈
        \item 最大化收益函数;另一个最小化支付函数。
        \item 每个时刻,每个玩家的控制决策都取决于另一个人之前做的决策。
    \end{itemize}
    \begin{definition}
    给定时间$0\le t \le T$ 和集合$\vA\subset\R^m, \vB\subset\R^l$,函数$\vf:\R^n\times \vA \times \vB \rightarrow R^n$。设可测映射$\valpha(\cdot): [t,T]\rightarrow \vA $为玩家1的控制函数,可测映射$\vbeta(\cdot): [t,T]\rightarrow \vB $为玩家2的控制函数,
    相应的动态方程为:
    \begin{align*}
        \Dot{\vx}(s)=\vf(\vx(s),\valpha(s),\vbeta(s))\ \ (t\le s \le T)
    \end{align*}
    \end{definition}
    
    \begin{definition}[支付(报酬)函数]
    博弈的支付函数定义为
    \begin{align}
        P_{\vx,t}[\valpha(\cdot),\vbeta(\cdot)]:=\int_t^T \gamma (\vx(s),\valpha (s),\vbeta(s)) \,ds +g(\vx(T)),
    \end{align}
    其中,玩家1通过调整$\valpha(\cdot)$最大化$P$, 玩家2通过调整$\vbeta(\cdot)$最小化$P$。
    \end{definition}

\end{frame}

\begin{frame}{\subsecname}
    \begin{definition}[博弈的控制集]     
    博弈的控制集定义为
    \begin{align*}
        \vA(t):=\{\valpha(\cdot):[t,T]\rightarrow A, ~\valpha(\cdot)\text{可测}\}, \\
        \vB(t):=\{\vbeta(\cdot):[t,T]\rightarrow B, ~\valpha(\cdot)\text{可测}\}.
    \end{align*}
    \end{definition}
    
    \begin{definition}[策略与策略集]
    \begin{itemize}
    \item 称映射$\vPhi : \vB(t)\rightarrow \vA(t)$为玩家1的\red{策略},如果对任意$s\in[t, T]$,有
        $$
        \vbeta(\tau) \equiv \hat{\vbeta}(\tau)  
        ~~\Rightarrow~~
        \vPhi[\vbeta](\tau) \equiv\vPhi [ \hat{\vbeta}](\tau), ~~\forall\tau \in [t,s].
        $$
    \item 称映射$\vPsi : \vA(t)\rightarrow \vB(t)$为玩家2的\red{策略},如果对任意$s\in[t, T]$,有
        $$
        \valpha(\tau) \equiv \hat{\valpha}(\tau)  
        ~~\Rightarrow~~
        \vPsi[\valpha](\tau) \equiv\vPsi [ \hat{\valpha}](\tau), ~~\forall\tau \in [t,s].
        $$ 
    \end{itemize}
    \end{definition}
    
    \begin{alertblock}{注}
    $\vPhi[\vbeta]$可看做是玩家1对玩家2选择控制$\vbeta(\cdot)$的反应,$\vPhi[\vbeta](\tau) \equiv\vPhi [ \hat{\vbeta}](\tau)$表示玩家1无法预见未来。
    \end{alertblock}
\end{frame}

\begin{frame}{\subsecname}
    \begin{definition}[策略集]
    策略集定义为
    \begin{align*}
    \cA(t):=\text{玩家1始于$t$时刻的策略集}, \quad 
    \cB(t):=\text{玩家2始于$t$时刻的策略集}.
    \end{align*}
    \end{definition}
    
    \begin{definition}[下值函数]
    \begin{equation}\label{eq:lower_value}
    v(\vx,t):=\inf_{\vPsi \in \cB(t)} \sup_{\valpha(\cdot)\in \vA(t)} P_{\vx,t} \left[\valpha(\cdot),\vPsi[\valpha](\cdot)\right].    
    \end{equation}
    \end{definition}
     
    \begin{definition}[上值函数]
    \begin{equation}\label{eq:upper_value}
    u(x,t):=\sup_{\vPhi \in \cA(t)} \inf_{\vbeta(\cdot)\in \vB(t)} P_{\vx,t} [\vPhi[\vbeta](\cdot),\vbeta(\cdot)]
    \end{equation}
    \end{definition}
      
    \begin{alertblock}{注}
      对于两个玩家,一个选择控制,另一个给出其策略加以回应。显然,选择策略的玩家更具优势。事实上,可以证明
      \begin{align*}
         v(x,t) \le u(x,t).
      \end{align*}
    \end{alertblock}
\end{frame}


\subsection{动态规划,Issac方程}
\begin{frame}{\secname}  
  \begin{theorem}[上值函数和下值函数的偏微分方程]
  设由\eqref{eq:lower_value}和\eqref{eq:upper_value}定义的$u,v$连续可导,则$u$是上Issac方程
  \begin{equation}\label{eq:upper_issac}
  \left\{
  \begin{array}{l}
    \ds u_t + \min_{\vbeta\in\vB}\max_{\valpha\in\vA}\left\{\vf(\vx,\valpha,\vbeta)\cdot\nabla_{\vx}u(\vx,t)+r(\vx,\valpha,\vbeta)\right\}= 0, \\[.1in]
    u(\vx, T) = g(\vx). 
  \end{array}
  \right.
  \end{equation}
  的解;$v$是下Issac方程
  \begin{equation}\label{eq:lower_issac}
  \left\{
  \begin{array}{l}
    \ds v_t + \max_{\valpha\in\vA}\min_{\vbeta\in\vB}\left\{\vf(\vx,\valpha,\vbeta)\cdot\nabla_{\vx}v(\vx,t)+r(\vx,\valpha,\vbeta)\right\}= 0, \\[.1in]
    v(\vx, T) = g(\vx). 
  \end{array}
  \right.
  \end{equation}
  的解。
    \end{theorem}
\end{frame}

\subsection{博弈与庞特里亚金极大值原理}
\begin{frame}{\secname}
  
   
\end{frame}  
  
\subsection{应用}
\begin{frame}{\secname}
\begin{example}[消耗战和进攻战]
    设两对手I和II正在交战,$x^1(t)$为I的资源供给,$x^2(t)$为II的资源供给。
\end{example}
\end{frame}

